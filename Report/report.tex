\documentclass{article}

\usepackage{tikz} 
\usetikzlibrary{automata, positioning, arrows} 

\usepackage{amsthm}
\usepackage{amsfonts}
\usepackage{amsmath}
\usepackage{amssymb}
\usepackage{fullpage}
\usepackage{color}
\usepackage{parskip}
\usepackage{hyperref}
  \hypersetup{
    colorlinks = true,
    urlcolor = blue,       % color of external links using \href
    linkcolor= blue,       % color of internal links 
    citecolor= blue,       % color of links to bibliography
    filecolor= blue,        % color of file links
    }
    
\usepackage{listings}
\usepackage[utf8]{inputenc}                                                    
\usepackage[T1]{fontenc}                                                       

\definecolor{dkgreen}{rgb}{0,0.6,0}
\definecolor{gray}{rgb}{0.5,0.5,0.5}
\definecolor{mauve}{rgb}{0.58,0,0.82}

\lstset{frame=tb,
  language=haskell,
  aboveskip=3mm,
  belowskip=3mm,
  showstringspaces=false,
  columns=flexible,
  basicstyle={\small\ttfamily},
  numbers=none,
  numberstyle=\tiny\color{gray},
  keywordstyle=\color{blue},
  commentstyle=\color{dkgreen},
  stringstyle=\color{mauve},
  breaklines=true,
  breakatwhitespace=true,
  tabsize=3
}

\newtheoremstyle{theorem}
  {\topsep}   % ABOVESPACE
  {\topsep}   % BELOWSPACE
  {\itshape\/}  % BODYFONT
  {0pt}       % INDENT (empty value is the same as 0pt)
  {\bfseries} % HEADFONT
  {.}         % HEADPUNCT
  {5pt plus 1pt minus 1pt} % HEADSPACE
  {}          % CUSTOM-HEAD-SPEC
\theoremstyle{theorem} 
   \newtheorem{theorem}{Theorem}[section]
   \newtheorem{corollary}[theorem]{Corollary}
   \newtheorem{lemma}[theorem]{Lemma}
   \newtheorem{proposition}[theorem]{Proposition}
\theoremstyle{definition}
   \newtheorem{definition}[theorem]{Definition}
   \newtheorem{example}[theorem]{Example}
\theoremstyle{remark}    
  \newtheorem{remark}[theorem]{Remark}

\title{CPSC-406 Report}
\author{Savana Chou  \\ Chapman University}

\date{\today} 

\begin{document}

\maketitle

\begin{abstract}
\end{abstract}

\setcounter{tocdepth}{3}
\tableofcontents

\section{Introduction}\label{intro}

\section{Week by Week}\label{homework}

\subsection{Week 1}

\section*{Exercise 1}

\subsection*{1. Word acceptance table}

\begin{center}
\begin{tabular}{c|c|c}
$w$ & accepted by $A_1$? & accepted by $A_2$? \\
\hline
$aaa$ & No & Yes \\
$aab$ & Yes & No \\
$aba$ & No & No \\
$abb$ & No & No \\
$baa$ & No & Yes \\
$bab$ & No & No \\
$bba$ & No & No \\
$bbb$ & No & No \\
\end{tabular}
\end{center}

\subsection*{2. Description of the languages}

\paragraph{Language of $A_1$.}

In $A_1$, any word beginning with $b$ immediately goes to a trap state.
If the word begins with $a$, the automaton stays in state $2$ while
reading $a$'s, and moves to the accepting state $3$ upon reading a
single $b$. Any further symbol leaves the accepting state.

Therefore, a word is accepted if and only if it consists of one or more
$a$'s followed by exactly one $b$.

\[
L(A_1) = \{ a^n b \mid n \ge 1 \}
\]

Equivalently, as a regular expression:

\[
L(A_1) = a^+ b
\]

\paragraph{Language of $A_2$.}

In $A_2$, the only accepting state is $3$. The automaton reaches state
$3$ precisely after reading two consecutive $a$'s. Any $b$ resets the
automaton to state $1$.

Thus, a word is accepted if and only if it ends with at least two
consecutive $a$'s.

\[
L(A_2) = \{ w \in \{a,b\}^* \mid w \text{ ends with } aa \}
\]

Equivalently, as a regular expression:

\[
L(A_2) = (a|b)^* aa
\]

\section*{Exercise 2}

Alphabet: $\Sigma = \{a,b\}$.

\subsection*{1. DFAs for the given languages}

\paragraph{(1) All words that end with \texttt{ab}.}

Let $M_1 = (Q,\Sigma,\delta,q_0,F)$ where

\[
Q = \{q_0,q_1,q_2\}, \quad q_0 \text{ start}, \quad F = \{q_2\}.
\]

Transitions:

\[
\begin{array}{c|cc}
 & a & b \\
\hline
q_0 & q_1 & q_0 \\
q_1 & q_1 & q_2 \\
q_2 & q_1 & q_0
\end{array}
\]

State $q_2$ represents that the word currently ends in \texttt{ab}.

\bigskip

\paragraph{(2) All words that contain \texttt{aba}.}

Let $M_2 = (Q,\Sigma,\delta,q_0,F)$ where

\[
Q = \{q_0,q_1,q_2,q_3\}, \quad q_0 \text{ start}, \quad F = \{q_3\}.
\]

Transitions:

\[
\begin{array}{c|cc}
 & a & b \\
\hline
q_0 & q_1 & q_0 \\
q_1 & q_1 & q_2 \\
q_2 & q_3 & q_0 \\
q_3 & q_3 & q_3
\end{array}
\]

State $q_3$ means the substring \texttt{aba} has been seen.

\bigskip

\paragraph{(3) Odd number of \texttt{a}'s and odd number of \texttt{b}'s.}

Use four states representing parity:

\[
Q = \{(E,E),(E,O),(O,E),(O,O)\}
\]

(start state $(E,E)$).  
Accepting state: $(O,O)$.

Reading $a$ toggles the first component;  
reading $b$ toggles the second component.

\bigskip

\paragraph{(4) Even number of \texttt{a}'s and odd number of \texttt{b}'s.}

Same construction as (3).

Start state: $(E,E)$.  
Accepting state: $(E,O)$.

Transitions again toggle parity accordingly.

\bigskip

\paragraph{(5) Any three consecutive characters contain at least one \texttt{a}.}

This is equivalent to saying the word does \emph{not} contain \texttt{bbb}.

Use states counting consecutive $b$'s:

\[
Q = \{q_0,q_1,q_2,q_3\}
\]

\begin{itemize}
\item $q_0$: no recent $b$
\item $q_1$: one consecutive $b$
\item $q_2$: two consecutive $b$'s
\item $q_3$: three consecutive $b$'s (trap)
\end{itemize}

Start: $q_0$.

Accepting states: $\{q_0,q_1,q_2\}$.

Transitions:

- On $a$: go to $q_0$ from any state except $q_3$
- On $b$:
  \[
  q_0 \to q_1,\quad
  q_1 \to q_2,\quad
  q_2 \to q_3,\quad
  q_3 \to q_3
  \]

\bigskip

\paragraph{(6) All words that contain \texttt{bbb}.}

Same structure as (5), but now:

Accepting state: $\{q_3\}$ only.

Once $q_3$ is reached, stay there.

\bigskip

\subsection*{2. Observations}

Several patterns appear:

\begin{itemize}
\item Languages involving substrings (e.g.\ \texttt{aba}, \texttt{bbb})
      use states that track partial progress toward matching the substring.
\item Parity conditions use a product construction (Cartesian product)
      of two 2-state automata, giving 4 states.
\item Conditions about avoiding a pattern (problem 5) are closely related
      to conditions about containing that pattern (problem 6);
      they differ mainly in which states are accepting.
\item Many DFAs follow a systematic design pattern:
      track exactly the minimal information necessary to decide acceptance.
\end{itemize}

\subsection*{Week 1 Discord Question}
Why is it important that the transition function of a DFA be total (defined for every state-symbol pair)? What would break, both formally and conceptually, if we allowed “missing” transitions?

\subsection{Week 2}

\section*{Exercise 1: Product Automata}

Let $\Sigma = \{a,b\}$.

\subsection*{1. Description of $L(\mathcal A^{(1)})$ and $L(\mathcal A^{(2)})$}

\subsubsection*{Language of $\mathcal A^{(1)}$}

In $\mathcal A^{(1)}$, states $2$ and $4$ are accepting, and state $3$ is a trap state with self-loops on both $a$ and $b$. 
From the transition structure we observe:

\begin{itemize}
    \item From state $2$, reading $a$ leads to the trap state.
    \item From state $4$, reading $b$ leads to the trap state.
\end{itemize}

Thus any occurrence of the substring $aa$ or $bb$ forces the automaton into the trap state.

Therefore,
\[
L(\mathcal A^{(1)}) 
= \{ w \in \{a,b\}^* \mid w \text{ contains no } aa \text{ and no } bb \}.
\]

Equivalently, $L(\mathcal A^{(1)})$ consists of all strings whose symbols strictly alternate.

\bigskip

\subsubsection*{Language of $\mathcal A^{(2)}$}

In $\mathcal A^{(2)}$, state $2$ is the only accepting state and state $3$ is a trap.

\begin{itemize}
    \item From the start state $1$, reading $b$ leads immediately to the trap.
    \item Thus every accepted string must begin with $a$.
    \item The automaton alternates between states $1$ and $2$ on every input symbol.
\end{itemize}

Since state $2$ is accepting, the automaton accepts precisely those strings that:
\begin{itemize}
    \item start with $a$, and
    \item have odd length.
\end{itemize}

Hence,
\[
L(\mathcal A^{(2)}) 
= \{ w \in \{a,b\}^* \mid w \text{ starts with } a \text{ and } |w| \text{ is odd} \}.
\]

\bigskip

\subsection*{2. Construction of the Intersection Automaton}

We construct the product automaton
\[
\mathcal A = \mathcal A^{(1)} \times \mathcal A^{(2)}.
\]

Its components are:

\begin{align*}
Q &= Q_1 \times Q_2, \\
q_0 &= (1,1), \\
F &= F_1 \times F_2.
\end{align*}

Since $F_1 = \{2,4\}$ and $F_2 = \{2\}$, we have
\[
F = \{(2,2), (4,2)\}.
\]

The transition function is defined by
\[
\delta((p,q), x) = (\delta_1(p,x), \delta_2(q,x))
\quad \text{for } x \in \{a,b\}.
\]

Keeping only reachable states, the automaton includes:
\[
(1,1),\ (2,2),\ (4,1),\ (3,1),\ (4,3),\ (3,3).
\]

Among these, the only reachable accepting state is
\[
(2,2).
\]

\bigskip

\subsection*{3. Correctness of the Product Construction}

By definition of the product construction, a string $w$ is accepted by $\mathcal A$ if and only if:

\begin{itemize}
    \item the run of $w$ in $\mathcal A^{(1)}$ ends in an accepting state, and
    \item the run of $w$ in $\mathcal A^{(2)}$ ends in an accepting state.
\end{itemize}

Thus,
\[
L(\mathcal A)
=
L(\mathcal A^{(1)}) \cap L(\mathcal A^{(2)}).
\]

\bigskip

\subsection*{4. Construction for the Union}

To obtain an automaton $\mathcal A'$ such that
\[
L(\mathcal A') = L(\mathcal A^{(1)}) \cup L(\mathcal A^{(2)}),
\]
we keep:

\begin{itemize}
    \item the same state set $Q_1 \times Q_2$,
    \item the same start state $(1,1)$,
    \item the same transition function.
\end{itemize}

We change only the accepting states to:

\[
F' = (F_1 \times Q_2) \cup (Q_1 \times F_2).
\]

That is, a state $(p,q)$ is accepting if either
\[
p \in F_1 \quad \text{or} \quad q \in F_2.
\]

\section*{Exercise 2: More Automata}

Let $\Sigma = \{a,b\}$.

\subsection*{1. Description of the Languages}

\subsubsection*{Language of $\mathcal B^{(1)}$}

The automaton $\mathcal B^{(1)}$ has states 
$\{p_0,p_1,p_2\}$ where $p_0$ is the start and only accepting state.
Each $b$ leaves the state unchanged, while each $a$ moves cyclically:
\[
p_0 \xrightarrow{a} p_1 \xrightarrow{a} p_2 \xrightarrow{a} p_0.
\]

Thus the automaton counts the number of $a$'s modulo $3$. Since only
$p_0$ is accepting, we obtain
\[
L(\mathcal B^{(1)}) 
= 
\{\, w \in \{a,b\}^* \mid \#_a(w) \equiv 0 \pmod{3} \,\}.
\]

In words: all strings whose number of $a$'s is divisible by $3$.

\bigskip

\subsubsection*{Language of $\mathcal B^{(2)}$}

The automaton $\mathcal B^{(2)}$ has states 
$\{q_0,q_1,q_2\}$ where $q_0$ is the start and only accepting state,
and $q_2$ is a trap state with loops on both $a$ and $b$.

The transitions show:
\[
q_0 \xrightarrow{a} q_1, 
\qquad
q_1 \xrightarrow{a} q_2.
\]

Thus any occurrence of two consecutive $a$'s sends the automaton
to the trap state $q_2$. Therefore,
\[
L(\mathcal B^{(2)}) 
=
\{\, w \in \{a,b\}^* \mid w \text{ does not contain } aa \,\}.
\]

\bigskip

\subsection*{2. Construction of the Intersection Automaton}

We construct the product automaton
\[
\mathcal B = \mathcal B^{(1)} \times \mathcal B^{(2)}.
\]

Its components are:
\begin{align*}
Q &= \{p_0,p_1,p_2\} \times \{q_0,q_1,q_2\}, \\
q_0^{\mathcal B} &= (p_0,q_0), \\
F &= \{(p_0,q_0)\}.
\end{align*}

The transition function is defined by
\[
\delta((p_i,q_j),x)
=
(\delta_1(p_i,x),\delta_2(q_j,x))
\quad \text{for } x \in \{a,b\}.
\]

Since both automata must accept, the only accepting state is
$(p_0,q_0)$.

\bigskip

\subsection*{3. Correctness of the Construction}

By definition of the product construction, a string $w$ is accepted by
$\mathcal B$ if and only if:

\begin{itemize}
    \item the run of $w$ in $\mathcal B^{(1)}$ ends in $p_0$, and
    \item the run of $w$ in $\mathcal B^{(2)}$ ends in $q_0$.
\end{itemize}

Hence,
\[
L(\mathcal B)
=
L(\mathcal B^{(1)})
\cap
L(\mathcal B^{(2)}).
\]

\bigskip

\subsection*{4. Construction of $\mathcal B'$ for the Union with a Complement}

We want an automaton $\mathcal B'$ such that
\[
L(\mathcal B')
=
L(\mathcal B^{(1)})
\cup
\overline{L(\mathcal B^{(2)})}.
\]

\paragraph{Step 1: Complement $\mathcal B^{(2)}$.}

Since $\mathcal B^{(2)}$ is a complete DFA, we obtain
$\overline{\mathcal B^{(2)}}$ by swapping accepting and non-accepting states.
Thus $q_1$ and $q_2$ become accepting, and $q_0$ becomes rejecting.

\paragraph{Step 2: Product Construction for the Union.}

We keep:

\begin{itemize}
    \item the same state set $Q_1 \times Q_2$,
    \item the same start state $(p_0,q_0)$,
    \item the same transition function.
\end{itemize}

We change the accepting states to:
\[
F'
=
(F_1 \times Q_2)
\cup
(Q_1 \times F_{\overline{2}}).
\]

That is, a state $(p,q)$ is accepting if either
\[
p = p_0
\quad \text{or} \quad
q \in \{q_1,q_2\}.
\]

\subsection*{Week 2 Discord Question}
In both exercises, we constructed product automata to recognize intersections and modified acceptance conditions to recognize unions or complements. Suppose instead we were given only one of the product automata (without being told it was constructed from two smaller DFAs). How could we determine whether this automaton can be decomposed into a product of two smaller DFAs?

\section{Synthesis}

\section{Evidence of Participation}

\section{Conclusion}\label{conclusion}

\begin{thebibliography}{99}
\bibitem[BLA]{bla} Author, \href{https://en.wikipedia.org/wiki/LaTeX}{Title}, Publisher, Year.
\end{thebibliography}

\end{document}
