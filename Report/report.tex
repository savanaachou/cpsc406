\documentclass{article}

\usepackage{tikz} 
\usetikzlibrary{automata, positioning, arrows} 

\usepackage{amsthm}
\usepackage{amsfonts}
\usepackage{amsmath}
\usepackage{amssymb}
\usepackage{fullpage}
\usepackage{color}
\usepackage{parskip}
\usepackage{hyperref}
  \hypersetup{
    colorlinks = true,
    urlcolor = blue,       % color of external links using \href
    linkcolor= blue,       % color of internal links 
    citecolor= blue,       % color of links to bibliography
    filecolor= blue,        % color of file links
    }
    
\usepackage{listings}
\usepackage[utf8]{inputenc}                                                    
\usepackage[T1]{fontenc}                                                       

\definecolor{dkgreen}{rgb}{0,0.6,0}
\definecolor{gray}{rgb}{0.5,0.5,0.5}
\definecolor{mauve}{rgb}{0.58,0,0.82}

\lstset{frame=tb,
  language=haskell,
  aboveskip=3mm,
  belowskip=3mm,
  showstringspaces=false,
  columns=flexible,
  basicstyle={\small\ttfamily},
  numbers=none,
  numberstyle=\tiny\color{gray},
  keywordstyle=\color{blue},
  commentstyle=\color{dkgreen},
  stringstyle=\color{mauve},
  breaklines=true,
  breakatwhitespace=true,
  tabsize=3
}

\newtheoremstyle{theorem}
  {\topsep}   % ABOVESPACE
  {\topsep}   % BELOWSPACE
  {\itshape\/}  % BODYFONT
  {0pt}       % INDENT (empty value is the same as 0pt)
  {\bfseries} % HEADFONT
  {.}         % HEADPUNCT
  {5pt plus 1pt minus 1pt} % HEADSPACE
  {}          % CUSTOM-HEAD-SPEC
\theoremstyle{theorem} 
   \newtheorem{theorem}{Theorem}[section]
   \newtheorem{corollary}[theorem]{Corollary}
   \newtheorem{lemma}[theorem]{Lemma}
   \newtheorem{proposition}[theorem]{Proposition}
\theoremstyle{definition}
   \newtheorem{definition}[theorem]{Definition}
   \newtheorem{example}[theorem]{Example}
\theoremstyle{remark}    
  \newtheorem{remark}[theorem]{Remark}

\title{CPSC-406 Report}
\author{Your Name  \\ Chapman University}

\date{\today} 

\begin{document}

\maketitle

\begin{abstract}
\end{abstract}

\setcounter{tocdepth}{3}
\tableofcontents

\section{Introduction}\label{intro}

\section{Week by Week}\label{homework}

\subsection{Week 1}

\section*{Exercise 1}

\subsection*{1. Word acceptance table}

\begin{center}
\begin{tabular}{c|c|c}
$w$ & accepted by $A_1$? & accepted by $A_2$? \\
\hline
$aaa$ & No & Yes \\
$aab$ & Yes & No \\
$aba$ & No & No \\
$abb$ & No & No \\
$baa$ & No & Yes \\
$bab$ & No & No \\
$bba$ & No & No \\
$bbb$ & No & No \\
\end{tabular}
\end{center}

\subsection*{2. Description of the languages}

\paragraph{Language of $A_1$.}

In $A_1$, any word beginning with $b$ immediately goes to a trap state.
If the word begins with $a$, the automaton stays in state $2$ while
reading $a$'s, and moves to the accepting state $3$ upon reading a
single $b$. Any further symbol leaves the accepting state.

Therefore, a word is accepted if and only if it consists of one or more
$a$'s followed by exactly one $b$.

\[
L(A_1) = \{ a^n b \mid n \ge 1 \}
\]

Equivalently, as a regular expression:

\[
L(A_1) = a^+ b
\]

\paragraph{Language of $A_2$.}

In $A_2$, the only accepting state is $3$. The automaton reaches state
$3$ precisely after reading two consecutive $a$'s. Any $b$ resets the
automaton to state $1$.

Thus, a word is accepted if and only if it ends with at least two
consecutive $a$'s.

\[
L(A_2) = \{ w \in \{a,b\}^* \mid w \text{ ends with } aa \}
\]

Equivalently, as a regular expression:

\[
L(A_2) = (a|b)^* aa
\]

\section*{Exercise 2}

Alphabet: $\Sigma = \{a,b\}$.

\subsection*{1. DFAs for the given languages}

\paragraph{(1) All words that end with \texttt{ab}.}

Let $M_1 = (Q,\Sigma,\delta,q_0,F)$ where

\[
Q = \{q_0,q_1,q_2\}, \quad q_0 \text{ start}, \quad F = \{q_2\}.
\]

Transitions:

\[
\begin{array}{c|cc}
 & a & b \\
\hline
q_0 & q_1 & q_0 \\
q_1 & q_1 & q_2 \\
q_2 & q_1 & q_0
\end{array}
\]

State $q_2$ represents that the word currently ends in \texttt{ab}.

\bigskip

\paragraph{(2) All words that contain \texttt{aba}.}

Let $M_2 = (Q,\Sigma,\delta,q_0,F)$ where

\[
Q = \{q_0,q_1,q_2,q_3\}, \quad q_0 \text{ start}, \quad F = \{q_3\}.
\]

Transitions:

\[
\begin{array}{c|cc}
 & a & b \\
\hline
q_0 & q_1 & q_0 \\
q_1 & q_1 & q_2 \\
q_2 & q_3 & q_0 \\
q_3 & q_3 & q_3
\end{array}
\]

State $q_3$ means the substring \texttt{aba} has been seen.

\bigskip

\paragraph{(3) Odd number of \texttt{a}'s and odd number of \texttt{b}'s.}

Use four states representing parity:

\[
Q = \{(E,E),(E,O),(O,E),(O,O)\}
\]

(start state $(E,E)$).  
Accepting state: $(O,O)$.

Reading $a$ toggles the first component;  
reading $b$ toggles the second component.

\bigskip

\paragraph{(4) Even number of \texttt{a}'s and odd number of \texttt{b}'s.}

Same construction as (3).

Start state: $(E,E)$.  
Accepting state: $(E,O)$.

Transitions again toggle parity accordingly.

\bigskip

\paragraph{(5) Any three consecutive characters contain at least one \texttt{a}.}

This is equivalent to saying the word does \emph{not} contain \texttt{bbb}.

Use states counting consecutive $b$'s:

\[
Q = \{q_0,q_1,q_2,q_3\}
\]

\begin{itemize}
\item $q_0$: no recent $b$
\item $q_1$: one consecutive $b$
\item $q_2$: two consecutive $b$'s
\item $q_3$: three consecutive $b$'s (trap)
\end{itemize}

Start: $q_0$.

Accepting states: $\{q_0,q_1,q_2\}$.

Transitions:

- On $a$: go to $q_0$ from any state except $q_3$
- On $b$:
  \[
  q_0 \to q_1,\quad
  q_1 \to q_2,\quad
  q_2 \to q_3,\quad
  q_3 \to q_3
  \]

\bigskip

\paragraph{(6) All words that contain \texttt{bbb}.}

Same structure as (5), but now:

Accepting state: $\{q_3\}$ only.

Once $q_3$ is reached, stay there.

\bigskip

\subsection*{2. Observations}

Several patterns appear:

\begin{itemize}
\item Languages involving substrings (e.g.\ \texttt{aba}, \texttt{bbb})
      use states that track partial progress toward matching the substring.
\item Parity conditions use a product construction (Cartesian product)
      of two 2-state automata, giving 4 states.
\item Conditions about avoiding a pattern (problem 5) are closely related
      to conditions about containing that pattern (problem 6);
      they differ mainly in which states are accepting.
\item Many DFAs follow a systematic design pattern:
      track exactly the minimal information necessary to decide acceptance.
\end{itemize}

\subsection*{Week 1 Discord Question}
Why is it important that the transition function of a DFA be total (defined for every state-symbol pair)? What would break, both formally and conceptually, if we allowed “missing” transitions?

\section{Synthesis}

\section{Evidence of Participation}

\section{Conclusion}\label{conclusion}

\begin{thebibliography}{99}
\bibitem[BLA]{bla} Author, \href{https://en.wikipedia.org/wiki/LaTeX}{Title}, Publisher, Year.
\end{thebibliography}

\end{document}
